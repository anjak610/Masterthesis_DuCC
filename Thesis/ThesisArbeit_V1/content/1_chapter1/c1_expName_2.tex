\chapter{Homographien in der Ebene}


(NOCH EINFÜGEN!!! HOMOGRAPHIEN BEINHALTEN INFORMATION ÜBER ROTATION UND TRANLSATION)

Im vorherigen Kapitel wurde ausführlich dargelegt, wie Koordinatensysteme ineinander überführt werden. Die folgenden Unterkapitel sollen anhand eines Beispiels zeigen, dass die in (Link Kapitel) gezeigten Transformationen in Matritzen zusammenfassen lassen. Im ersten Beispiel wird davon ausgegangen, dass die intrinsischen und extrinsischen Kameraparameter unbekannt sind und nur die Bildpunkte von beiden Kamerakoordinatensystemen bekannt sind. Des Weiteren wird festgelegt, dass sich alle Bildpunkte auf einer Ebene befinden und somit die selbe Tiefe $z$ besitzen. Die behauptung ist nun, dass es mgöich ist eine 3x3-Homographiematrix zu ermitteln, welche die Punkte von Kamera eins in die Punkte von Kamera zwei und umgekehrt überführen kann. Eine Homographie ist eine projektive Transformation zwischen zwei Ebenen. Dabei bleiben Kollinearitäten und die Reihenfolge von Punkten auf Geraden(z.B Schnittpunkte mit anderne Geraden) erhalten. Aufgrund der Ebenenannahme, kann solch eine projektive Transformation durch eine 3x3-Homograohiematrix ausgedrückt werden\cite{Roser}. Sprich die entstehende projektive Transformation projiziert jede Figur in eine Figur gleicher projektiver Entsprechung\cite{HZ}.Die Homographie ist eine allgemeine projektive Transformation, welche Abbildungen einer Ebene in eine andere beschreiben, dabei können Punkte in Punkte oder Gerade in Geraden überführt werden, wobei das Doppelverhältnis erhalten bleibt.Sind die Punkte $A',B',C'$ und $D'$ die projektiven Bilder eines Systems von vier kollinearen Punkten, so ist $(A',B',C',D') =(A,B,C,D)$\cite{Peiffer}. Schon Euler hatte Bewegungen untersucht, er hatte im Prinzip gezeigt, dass eine ebene Kongruenzabbildung eine Rotaion, eine Translation oder eine Translation gefolgt von einer Spiegelung ist. Möbius nahm den Eulerschen Terminus affine Transformation wieder auf, um solche Transformationen zu benennen, die die Parallelität erhalten, aber nicht abstandstreu sind. Die allgemeinste Transformation, die Möbius studierte, war die Homographie, die er Kollineation nannte\cite{Peiffer}\\



 
Es seien \ensuremath{x = \begin{pmatrix}
		x_1\\x_2\\x_3
\end{pmatrix}} die homogenen Koordinaten eines Punktes der projektiven Ebene und \ensuremath{x' = \begin{pmatrix}
x'_1\\x'_2\\x'_3
\end{pmatrix}} die Punkte des projektiv transformierten Punktes. Dann gilt

\begin{gather}
	x' = Hx\\
	Hx = \begin{bmatrix}
	{h_1}^T \cdot x\\{h_2}^T \cdot x\\{h_3}^T \cdot x
	\end{bmatrix} \\
	\leadsto 
	x'= Hx= \begin{bmatrix}
	h_{11}x_1+h_{12}x_2+h_{13x_3}\\
	h_{21}x_1+h_{22}x_2+h_{23x_3}\\
	h_{31}x_1+h_{32}x_2+h_{33x_3}
	\end{bmatrix}\\
	\leadsto 
	H=\begin{bmatrix}
	h_{11}&h_{12}&h_{13}\\
	h_{21}&h_{22}&h_{23}\\
	h_{31}&h_{32}&h_{33}
	\end{bmatrix}
\end{gather}

Dabei müssen die Koeffizienten so geartet sein, dass die zugehörige Transformation umkehrbar ist. \cite{HZ}\cite{Peiffer}.Sprich es muss gelten dass wenn 
\begin{gather}
	x'=Hx\\
	x= H^{-1}x'
\end{gather}\\

Danach wird dann die Beziehung von Punkten im Raum aufgezeigt, welche sich nicht auf einer Ebene befinden. Die Relationen dieser Punkte zueinander lassen sich mit der so genannten Epipolargeometrie beschreiben. Mehr dazu ab Kapitel (link zu Kapitel 3.3).




\section{Homographie zwischen der Abbildungen eines Quadrates einer definierten Ausgangskamera und einer um ihr Projektionszentrum rotierten Kamera}

In diesem Beispiel wird die Abbildung eines Quadrats einer Kamera in die einer anderen Kamera transformiert. Danach wird eine Homographiematrix ermittelt und auf ihre Gültigkeit hin untersucht. Sprich es wird geschaut, ob sich die Punkte der einen Kamera in die Punkte der anderen nur mit Hilfe dieser Matrix überführen lassen und anders herum. Erwähnt sei außerdem, dass in diesem Beispiel nicht von überbestimmten Systemen ausgegangen wird. Für diese gilt ein leicht anderer Algorithmus, welcher auch in den folgenden Algorithmen dieser Arbeit, wie zum Beispiel bei der Findung der Fundamentalmatrix gebraucht wird und dort genauer besprochen wird.
Die Kamerakoordinatensysteme unterscheiden sich durch eine Drehung um 180° um die \ensuremath{\hat{e}_3}-Achse. Der Ursprung beider Kamerakoordinatensysteme entspricht dem jeweiligen Projektionszentrum. Es werden zwei Bilder der selben Szene mit diesen Kameras aufgenommen. Die Behauptung ist, dass sich die beiden entstandenen Bilder mit Einer Homographie ineinander überführen lassen. 

\begin{gather}
H=
\begin{bmatrix}
h_{11}&h_{12}&h_{13}\\
h_{21}&h_{22}&h_{23}\\
h_{31}&h_{32}&h_{33}
\end{bmatrix}
\end{gather}\\

Eines der Kamerakoordinatensysteme wird in Abbildung 3.1 nochmal grafisch im Vergleich zum definierten Weltkoordinatensystem veranschaulicht.

\begin{minipage}{\linewidth}
	\centering
	\includegraphics[width=1.\linewidth]{images/Rotation.png}
	\captionof{figure}{Weltkoordinatensystem \ensuremath{K = (O,\hat{e}_1,\hat{e}_2,\hat{e}_3)} und Kamerakoordinatensystem \ensuremath{K_c=(O_c, \hat{c}_1, \hat{c}_2, \hat{c}_3)}.}
\end{minipage}\\ \\

Zur Übersichtlichkeit wurden die Koordinatenssysteme in Abbildung 3.1 verschoben voneinander dargestellt. In Wirklichkeit sind die Ursprünge \ensuremath{O} und \ensuremath{O_c} deckungsgleich. Als nächstes werden die intrinsischen Kameraparameter festgelegt, hierbei gilt: 

\begin{gather}
\leftidx{_{K_{c}}}{\begin{bmatrix}
	\pi
	\end{bmatrix}}{_{K_{c}}} 
= 
\begin{pmatrix}
\zeta&0&0&0\\
0&\zeta&0&0\\
0&0&\zeta&0\\
0&0&1&0
\end{pmatrix}\\
\leadsto
\begin{pmatrix}
X\\Y\\Z
\end{pmatrix} = 
\leftidx{_{K_{c}}}{\begin{bmatrix}
	\pi
	\end{bmatrix}}{_{K_{c}}}
\begin{pmatrix}
X\\Y\\Z\\1
\end{pmatrix}
=
\begin{pmatrix}
\zeta X\\\zeta Y\\\zeta Z\\Z
\end{pmatrix}
=
\begin{pmatrix}
\frac{\zeta}{Z} X\\\frac{\zeta}{Z} Y\\\zeta\\1
\end{pmatrix}
\end{gather}\\

Für das Beispiel gilt zunächst die Bedingung $\zeta \neq 0$. Des Weiteren soll gelten, dass  \ensuremath{\hat{c}_3} gleich der Lotgeraden vom Sender zum Projektionszentrum entsptricht und somit folgt, dass \ensuremath{\hat{c}_1} in Sensorebene liegt und \ensuremath{\hat{c}_2 = \hat{c}_3 \times \hat{c}_1} ist. Der Quader besteht aus den Punkten $A,B,C, D$ und $E$ in homogenen Weltkoordinaten. Das Koordinatensystem von Kamera eins ist gleich dem um \ensuremath{180^\circ} gedrehte Weltkoordinatensystem. Die Punkte in Weltkoordinaten lauten wie folgt. 

\begin{gather}
(A)_K=\begin{pmatrix}
0\\0\\2\\1
\end{pmatrix}, 
(B)_K=
\begin{pmatrix}
1\\0\\2\\1
\end{pmatrix},
(C)_K=
\begin{pmatrix}
1\\1\\2\\1
\end{pmatrix},
(D)_K=
\begin{pmatrix}
0\\1\\2\\1
\end{pmatrix},
(E)_K=
\begin{pmatrix}
\frac{1}{2}\\\frac{1}{2}\\2\\1
\end{pmatrix}
\end{gather}\\

Kamera zwei wird für die Aufnahme der Szene zusätzlich noch um \ensuremath{45^\circ}zu Kamera eins eingedreht.
Für die Transformation der Weltkoordinatentupel in Kamera eins und Kamera zwei müssen also zuerst die Punkte in die jeweiligen Kamerakoordinatensystemen umgerechnet werden. Für die Transformation der Weltkooridinaten in die entsprechenden Kamerakoordinaten werden also zwei Matrizen \ensuremath{T_1} und \ensuremath{T_2} benötigt. \ensuremath{T_1} bewirkt die Drehung der Punkte um 180° um die \ensuremath{\hat{e}_3}-Achse. Um \ensuremath{T_2} zu erhalten wird \ensuremath{T_1} zusätzlich noch mit einer weiteren Transformationsmatrix, welche die Drehung um \ensuremath{45^\circ} beinhaltet, verrechnet. Die so erhaltenen Matrizen \ensuremath{T_1} und \ensuremath{T_2} können nun dazu verwendet werden, die Punkte Bezüglich des Weltkoordinatensystems in Punkte bezüglich der jeweiligen Kamerakoordinatensysteme zu transformieren. Im ersten Schritt erfolgt die Drehung um \ensuremath{180^\circ} um die \ensuremath{\hat{e}_3}-Achse.


\begin{gather}
	\begin{bmatrix}
		\cos(180)&-\sin(180)&0&0\\
		\sin(180)&\cos(180)&0&0\\
		0&0&1&0\\
		0&0&0&1
	\end{bmatrix}
	\begin{pmatrix}
		\hat{e}_1,\hat{e}_2,\hat{e}_3,O
	\end{pmatrix}
	=
	\begin{bmatrix}
		-1&0&0&0\\
		0&-1&0&0\\
		0&0&1&0\\
		0&0&0&1
	\end{bmatrix} 
	\begin{pmatrix}
		\hat{e}_1,\hat{e}_2,\hat{e}_3,O
	\end{pmatrix}\\
	\begin{bmatrix}
		-1&0&0&0\\
		0&-1&0&0\\
		0&0&1&0\\
		0&0&0&1
	\end{bmatrix}^T =\begin{bmatrix}
	-1&0&0&0\\
	0&-1&0&0\\
	0&0&1&0\\
	0&0&0&1
	\end{bmatrix}= T_1
\end{gather}\\

Im nächsten Schritt wird die Kamera um \ensuremath{45^\circ} zu Kamera eins eingedreht, es ergibt sich die folgende Matrix \ensuremath{T_2}:

\begin{gather}
	\cos(45)=\sin(45)=\frac{1}{\sqrt{2}} = a\\
	\begin{bmatrix}
		a&0&a&0\\
		0&1&0&0\\
		-a&0&a&0\\
		0&0&0&1
	\end{bmatrix}
	\begin{bmatrix}
		-1&0&0&0\\
		0&-1&0&0\\
		0&0&1&0\\
		0&0&0&1
	\end{bmatrix}
	=\begin{bmatrix}
		-a&0&a&0\\
		0&-1&0&0\\
		a&0&a&0\\
		0&0&0&1
	\end{bmatrix}
	\begin{pmatrix}
		\hat{e}_1,\hat{e}_2,\hat{e}_3,O
	\end{pmatrix}\\
	=\begin{bmatrix}
	-a&0&a&0\\
	0&-1&0&0\\
	a&0&a&0\\
	0&0&0&1
	\end{bmatrix}^T
	=
	\begin{bmatrix}
		-a&0&a&0\\
		0&-1&0&0\\
		a&0&a&0\\
		0&0&0&1
	\end{bmatrix}
	= T_2
\end{gather}\\

Handelt es sich bei den Kamerakoordinatensystemen nicht um karteische Koordinatensysteme, so muss von den Rotationsmatrizen jeweil die Inverse genommen werden, um \ensuremath{T_1} und \ensuremath{T_2} zu erhalten. Sind die Matrizen \ensuremath{T_1} und \ensuremath{T_2} ermittelt, können nun die Weltkoordinatentupel in die Koordinatentupel von Kamera eins und Kamera zwei transformiert werden. Zunächst die transformation der Punkte in die um \ensuremath{180^\circ} gedrehte Kamera eins.

\begin{gather}
	\begin{pmatrix}
		(A)_{K_{c1}}\\1
	\end{pmatrix}
	=
	\begin{bmatrix}
		-1&0&0&0\\
		0&-1&0&0\\
		0&0&1&0\\
		0&0&0&1
	\end{bmatrix}
	\begin{pmatrix}
		0\\0\\2\\1
	\end{pmatrix}
	=
	\begin{pmatrix}
		0\\0\\2\\1
	\end{pmatrix}\\
	\begin{pmatrix}
		(B)_{K_{c1}}\\1
	\end{pmatrix}
	=
	\begin{bmatrix}
		-1&0&0&0\\
		0&-1&0&0\\
		0&0&1&0\\
		0&0&0&1
	\end{bmatrix}
	\begin{pmatrix}
		1\\0\\2\\1
	\end{pmatrix}
	=
	\begin{pmatrix}
		-1\\0\\2\\1
	\end{pmatrix}\\
	\begin{pmatrix}
		(C)_{K_{c1}}\\1
	\end{pmatrix}
	=
	\begin{bmatrix}
		-1&0&0&0\\
		0&-1&0&0\\
		0&0&1&0\\
		0&0&0&1
	\end{bmatrix}
	\begin{pmatrix}
		1\\1\\2\\1
	\end{pmatrix}
	=
	\begin{pmatrix}
		-1\\-1\\2\\1
	\end{pmatrix}\\
	\begin{pmatrix}
		(D)_{K_{c1}}\\1
	\end{pmatrix}
	=
	\begin{bmatrix}
		-1&0&0&0\\
		0&-1&0&0\\
		0&0&1&0\\
		0&0&0&1
	\end{bmatrix}
	\begin{pmatrix}
		0\\1\\2\\1
	\end{pmatrix}
	=
	\begin{pmatrix}
		0\\-1\\2\\1
	\end{pmatrix}\\
	\begin{pmatrix}
	(E)_{K_{c1}}\\1
\end{pmatrix}
=
\begin{bmatrix}
	-1&0&0&0\\
	0&-1&0&0\\
	0&0&1&0\\
	0&0&0&1
\end{bmatrix}
\begin{pmatrix}
	\frac{1}{2}\\\frac{1}{2}\\2\\1
\end{pmatrix}
=
\begin{pmatrix}
	-\frac{1}{2}\\-\frac{1}{2}\\2\\1
\end{pmatrix}\\
\end{gather}

Danach folgt die Transformation der Weltkoordinaten in die \ensuremath{180^\circ} um die \ensuremath{\hat{e}_3}-Achse und zusätzlich \ensuremath{45^\circ} um die neu entstandene \ensuremath{\hat{c}_2}-Achse gedrehte Kamera zwei mit Matrix \ensuremath{T_2}.

\begin{gather}
	\begin{pmatrix}
		(A)_{K_{c2}}\\1
	\end{pmatrix}
	=
	\begin{bmatrix}
		-a&0&a&0\\
		0&-1&0&0\\
		a&0&a&0\\
		0&0&0&1
	\end{bmatrix}
	\begin{pmatrix}
		0\\0\\2\\1
	\end{pmatrix}
	=
	\begin{pmatrix}
		2a\\0\\2a\\1
	\end{pmatrix}\\
	\begin{pmatrix}
		(B)_{K_{c2}}\\1
	\end{pmatrix}
	=
	\begin{bmatrix}
		-a&0&a&0\\
		0&-1&0&0\\
		a&0&a&0\\
		0&0&0&1
	\end{bmatrix}
	\begin{pmatrix}
		1\\0\\2\\1
	\end{pmatrix}
	=
	\begin{pmatrix}
		a\\0\\3a\\1
	\end{pmatrix}\\
	\begin{pmatrix}
		(C)_{K_{c2}}\\1
	\end{pmatrix}
	=
	\begin{bmatrix}
		-a&0&a&0\\
		0&-1&0&0\\
		a&0&a&0\\
		0&0&0&1
	\end{bmatrix}
	\begin{pmatrix}
		1\\1\\2\\1
	\end{pmatrix}
	=
	\begin{pmatrix}
		a\\-1\\3a\\1
	\end{pmatrix}\\
	\begin{pmatrix}
		(D)_{K_{c2}}\\1
	\end{pmatrix}
	=
	\begin{bmatrix}
		-a&0&a&0\\
		0&-1&0&0\\
		a&0&a&0\\
		0&0&0&1
	\end{bmatrix}
	\begin{pmatrix}
		0\\1\\2\\1
	\end{pmatrix}
	=
	\begin{pmatrix}
		2a\\-1\\2a\\1
	\end{pmatrix}\\
	\begin{pmatrix}
	(E)_{K_{c2}}\\1
\end{pmatrix}
=
\begin{bmatrix}
	-a&0&a&0\\
	0&-1&0&0\\
	a&0&a&0\\
	0&0&0&1
\end{bmatrix}
\begin{pmatrix}
	\frac{1}{2}\\\frac{1}{2}\\2\\1
\end{pmatrix}
=
\begin{pmatrix}
	\frac{3}{2}a\\-\frac{1}{2}\\\frac{5}{2}a\\1
\end{pmatrix}
\end{gather}\\

Danach müssen die entstandenen Koordinatentupel von Kamera eins und Kamera zwei noch mit ihren jeweiligen Projektionsmatritzen verrechnet werden. $\zeta$ bekommt in diesem Beispiel den Wert -1, das bedeutet laut der Definition der Kamerakoordinatensysteme, dass der Sensor sich hinter dem Projektionszentrum befindet. Das entstehende Bild ist somit um \ensuremath{180^\circ} gedreht auf dem Sensor abgebildet.

\begin{gather}
	\leftidx{_{K_{c1}}}{\begin{bmatrix}
			\pi
	\end{bmatrix}}{_{K_{c1}}}
	=		\leftidx{_{K_{c2}}}{\begin{bmatrix}
			\pi
	\end{bmatrix}}{_{K_{c2}}}
	=
	\begin{pmatrix}
		\zeta&0&0&0\\
		0&\zeta&0&0\\
		0&0&\zeta&0\\
		0&0&1&0
	\end{pmatrix}=
	\begin{pmatrix}
		-1&0&0&0\\
		0&-1&0&0\\
		0&0&-1&0\\
		0&0&1&0
	\end{pmatrix}
\end{gather}

Die fünf Punkte werden jeweils in einer Matrix zusammengeschrieben, so entstehen folgende Koordinaten für Kamera eins.

\begin{gather}
	\begin{pmatrix}
		-1&0&0&0\\
		0&-1&0&0\\
		0&0&-1&0\\
		0&0&1&0
	\end{pmatrix}
	\begin{pmatrix}
		0&-1&-1&0&-\frac{1}{2}\\
		0&0&-1&-1&-\frac{1}{2}\\
		2&2&2&2&2\\
		1&1&1&1&1
	\end{pmatrix}=
	\begin{pmatrix}
		0&1&1&0&-\frac{1}{2}\\
		0&0&1&1&-\frac{1}{2}\\
		-2&-2&-2&-2&-2\\
		2&2&2&2&2
	\end{pmatrix}\\
	\simeq
	\begin{pmatrix}
		0&\frac{1}{2}&\frac{1}{2}&0&\frac{1}{4}\\
		0&0&\frac{1}{2}&\frac{1}{2}&\frac{1}{4}\\
		-1&-1&-1&-1&-1\\
		1&1&1&1&1
	\end{pmatrix}
\end{gather}
\pagebreak

Und für Kamera zwei erhalten wir dann dem entsprechend folgendes:
(HIER NOCH PUNKT E EINFÜGEN)

\begin{gather}
	\begin{pmatrix}
		-1&0&0&0\\
		0&-1&0&0\\
		0&0&-1&0\\
		0&0&1&0
	\end{pmatrix}
	\begin{pmatrix}
		2a&a&a&2a&\frac{3}{2}a\\
		0&0&-1&-1&-\frac{1}{2}\\
		2a&3a&3a&2a&\frac{5}{2}a\\
		1&1&1&1&1
	\end{pmatrix}=
	\begin{pmatrix}
		-2a&-a&-a&-2a&-\frac{3}{2}a\\
		0&0&1&1&\frac{1}{2}\\
		-2a&-3a&-3a&-2a&-\frac{5}{2}a\\
		2a&3a&3a&2a&\frac{5}{2}a
	\end{pmatrix}\\
	\simeq
	\begin{pmatrix}
		-1&-\frac{1}{3}&-\frac{1}{3}&-1&\frac{3}{5}\\
		0&0&\frac{1}{3a}&\frac{1}{2a}&\frac{1}{5}a\\
		-1&-1&-1&-1&-1\\
		1&1&1&1&1
	\end{pmatrix}
\end{gather}\\

Die Entstandenen Punkte beider Kameras wurden in Geogebra eingegeben und das Programm liefert die folgende beide Quadrate\\

\begin{minipage}{\linewidth}
	\centering
	\includegraphics[width=1.\linewidth]{images/Geogebra.png}
	\captionof{figure}{in blau ist die Abblidung des Quaders von Kamera eins und in rot die Abbildung des selben Quaders in Kamera zwei}
\end{minipage}\\ \\

Die ermittelten Punkte sollen nun durch eine eigens aufgestellte Homographiematrix ineinander überführt werden. Um eine Homographiematrix mit $
H=
\begin{bmatrix}
h_{11}&h_{12}&h_{13}\\
h_{21}&h_{22}&h_{23}\\
h_{31}&h_{32}&h_{33}
\end{bmatrix}
$ zu ermitteln werden die Punkte beider Kameras in eine Koeffizientenmatrix eingetragen, welche sich nach folgendem Schema ergibt:

\begin{gather}
\begin{bmatrix}
h_{11}&h_{12}&h_{13}\\
h_{21}&h_{22}&h_{23}\\
h_{31}&h_{32}&h_{33}
\end{bmatrix}
\cdot
\begin{bmatrix}
\\P_{K1}\\\\
\end{bmatrix}
=
\begin{bmatrix}
\\P_{K2}\\\\
\end{bmatrix}\\
\begin{bmatrix}
h_{11}&h_{12}&h_{13}\\
h_{21}&h_{22}&h_{23}\\
h_{31}&h_{32}&h_{33}
\end{bmatrix}
\cdot
\begin{bmatrix}
x\\y\\z
\end{bmatrix}
=
\begin{bmatrix}
x'\\y'\\z'
\end{bmatrix}\\
\end{gather}

Hieraus ergeben sich die folgenden drei Gleichungssysteme

\begin{gather}
h_{11}x+h_{12}y+h_{13}z= \lambda x'\\
h_{21}x+h_{22}y+h_{23}z= \lambda y'\\
h_{31}x+h_{32}y+h_{33}z= \lambda z'
\end{gather}

Da mit homogenen Koordinaten gearbeitet wird und somit $z$ und $z'$ = 1 sind, ergibt sich für die letzte Zeile $h_{31}x+h_{32}y+h_{33}z= 1$. Dieser Ausdruck wird in den ersten beiden Gleichungen für $\lambda$ eingesetzt. Es ergeben sich pro Punktepaar jeweils zwei Gleichungen. 

\begin{gather}
	h_{11}x+h_{12}y+h_{13}z= (h_{31}x+h_{32}y+h_{33}z) \cdot x'\\
		h_{21}x+h_{22}y+h_{23}z= (h_{31}x+h_{32}y+h_{33}z) \cdot y'
\end{gather}

Für den Aufbau der Koeffizientenmatrix werden beide Ausdrücke noch nach Null aufgelöst

\begin{gather}
	h_{11}x+h_{12}y+h_{13}z -(h_{31}x+h_{32}y+h_{33}z) \cdot x'= 0 \\	h_{21}x+h_{22}y+h_{23}z-(h_{31}x+h_{32}y+h_{33}z) \cdot y'=0
	\end{gather}
\begin{gather}
	\leadsto h_{11}x+h_{12}y+h_{13}z -h_{31}x\cdot x' - h_{32}y \cdot x'-h_{33}z\cdot x'= 0\\
	\leadsto h_{21}x+h_{22}y+h_{23}z-h_{31}x\cdot y -h_{32}y \cdot y -h_{33}z) \cdot y'=0
\end{gather}

Für die Koeffizientenmatrix ergibt sich dann folgendes.

\begin{gather}
	\begin{pmatrix}
	x_1&y_1&1&0&0&0&x_1 x'_1&y_1 x'_1 & 1\cdot x'_1\\
	0&0&0&x_1&y_1&1&x_1 y'_1&y_1 y'_1 & 1\cdot y'_1\\
	&&&&&.&&&\\	
	&&&&&.&&&\\	
	&&&&&.&&&\\	
	x_i&y_i&1&0&0&0&x_i x'_i&y_i x'_i & 1\cdot x'_i\\
	0&0&0&x_i&y_i&1&x_i y'_i&y_i y'_i & 1\cdot y'_i
	\end{pmatrix}
	\cdot
	\begin{pmatrix}
	h1\\h2\\.\\.\\.\\hi
	\end{pmatrix}
	=0
\end{gather}

Wenn der Idealfall gilt, sprich wenn der Rang der Koeffizientenmatrix genau acht beträgt, kann aus der Koeffizientenmatrix einfach der Nullraum berechnet werden. Dieser Nullraum entspricht dem Kern der Koeffizientenmatrix und liefert die neun Einträge der Homographiematrix. Das Ergebnis ist also ein Spaltenvektor mit 9 Einträgen, welcher in die 3x3-Homographiematrix eingetragen werden kann. \cite{HZ}\cite{Schwarz}. Tritt nun der Fall ein, dass man ein überbestimmtes System besitzt, was Beipielsweise auftritt wenn mehr als neun Punktepaare durch eine Homographie ineinander überführt werden sollen, so kann nicht mehr der Nullram für die Berechnung der Homographiematrix benutzt werden. Das resultierende Ergebnis würde zwei oder mehr voneinander unabhängige Lösungen bieten, aus denen man die reale Lösung erst noch herausfinden muss. Mehr zu diesem Verfahren gibt es bei der Berechnung der Fundamentalmatrix.\cite{HZ} Für die Lösung überbestimmter Gleichungssysteme bietet sich die Singulärwertzerlegung an\cite{HZ}\cite{Scholz}. Das bedeutet es wird nicht derjenige Vektor $x$ gesucht für den gilt $H \cdot x = 0$, sondern es wird derjenige Vektor $x$ gesucht, für den \ensuremath{\parallel H \cdot x\parallel} minimal wird\cite{HZ}. Die Singulärwertzerlegung von $A$ ist eine Faktorisierung einer beliebeigen Matrix \ensuremath{A \in \mathbb{R}^{mxn}} der Form \ensuremath{A = U \cdot S \cdot V^T} mit orthogonalen Matrizen \ensuremath{U \in \mathbb{R}^{m \times n}} und \ensuremath{V \in \mathbb{R}^{m \times n}} sowie mit einer Diagonalmatrix. 

\begin{gather}
	S = \begin{pmatrix}
	s_1&&...&&0&0&&...&&0\\
	.&.&&&.&.&&&&.\\
	.&&.&&.&.&&&&.\\
	.&&&.&.&.&&&&.\\
	0&&...&&s_r&0&&...&&0\\	
	0&&...&&0&0&&...&&0\\
	.&&&&.&.&&&&.\\
	.&&&&.&.&&&&.\\	
	.&&&&.&.&&&&.\\	
	0&&...&&0&0&&...&&0\\	
	\end{pmatrix}
\end{gather}

Dabei gelte \ensuremath{s_1 \geq s_2 \geq ... \geq s_r \ge 0 }, und die Zahlen $s_1$ bis $s_r$ werden als Singulärwerte von $A$ bezeichnet\cite{Scholz}. Entsprechend dieser Methode wird eine Singulärwertszerlegung, kurz $SVD$ der entstandenen Koeffizientenmatrix, welche gleich der Form von Matrix $A$ und im fortlaufenden auch mit $A$ bezeichnet wird, durchgeführt. Wir erhalten 3 Matrizen $U \cdot S\cdot V^T$. Durch die Zerlegung sind die diagonaleinträge von $S$ in einer absteigenden Reihenfolge sortiert. Der kleinste Sigulärwert korrespondiert auf diese Weise mit der letzten Spalte von $V$. Somit gleichen die neun Einträge der Homographiematrix gleich der letzten Spalte von $V$. Das Ergebnis für $H$ hat dann die folgende Form

\begin{gather}
	H=
	\begin{pmatrix}
	v_{19}&v_{29}&v_{39}\\
	v_{49}&v_{59}&v_{69}\\
	v_{79}&v_{89}&v_{99}
	\end{pmatrix}
\end{gather}

Für das Minimalbeispiel mit reinen Punkten, welches in diesem Kapitel erstellt wurde, würde die Herleitung der Homographiematrix über die Ermittlung des Nullraumes der Koeffizientenmatrix genügen. Für die Matrix $H$ ergibt sich aus den Werten der im Beispiel verwendeten Punkte

\begin{gather}
	\begin{pmatrix}
	1&0&-1\\
	0&1.41421&0\\
	1&0&1
	\end{pmatrix}
\end{gather}

Nun werden die Punkte aus Kamera eins und die Punkte aus Kamera zwei jeweils mit Hilfe von $H$ ineinander überführt. Hierzu werden die Tiefenwerte $z$ der Punkte vernachlässigt. Die Punkte liegen alle auf einer Ebene, die Tiefe ist somit für die Umrechnung der Punkte nicht relevant.

\begin{gather}
		\begin{pmatrix}
	-1&-\frac{1}{3}&-\frac{1}{3}&-1\\
	0&0&\frac{1}{3a}&\frac{1}{2a}\\
	-1&-1&-1&-1\\
	1&1&1&1
	\end{pmatrix}
	=
		\begin{pmatrix}
	-1&-\frac{1}{3}&-\frac{1}{3}&-1\\
	0&0&\frac{1}{3a}&\frac{1}{2a}\\
	1&1&1&1
	\end{pmatrix}\\
		\begin{pmatrix}
	0&\frac{1}{2}&\frac{1}{2}&0\\
	0&0&\frac{1}{2}&\frac{1}{2}\\
	-1&-1&-1&-1\\
	1&1&1&1
	\end{pmatrix}
	=
		\begin{pmatrix}
	0&\frac{1}{2}&\frac{1}{2}&0\\
	0&0&\frac{1}{2}&\frac{1}{2}\\
	1&1&1&1
	\end{pmatrix}
\end{gather}

Probe:
\begin{gather}
x'=H \cdot x \leadsto 
	\begin{pmatrix}
	-1&-\frac{1}{3}&-\frac{1}{3}&-1\\
	0&0&\frac{1}{3a}&\frac{1}{2a}\\
	1&1&1&1
\end{pmatrix}= 	\begin{pmatrix}
1&0&-1\\
0&\frac{2}{\sqrt{2}}&0\\
1&0&1
\end{pmatrix}
\cdot
		\begin{pmatrix}
0&\frac{1}{2}&\frac{1}{2}&0\\
0&0&\frac{1}{2}&\frac{1}{2}\\
1&1&1&1
\end{pmatrix}
\end{gather}

und umgekehrt:

\begin{gather}
x=H^{-1} \cdot x' \leadsto 
\begin{pmatrix}
0&\frac{1}{2}&\frac{1}{2}&0\\
0&0&\frac{1}{2}&\frac{1}{2}\\
1&1&1&1
\end{pmatrix}
= 	\begin{pmatrix}
\frac{1}{2}&0&\frac{1}{2}\\
0&\frac{1}{\sqrt{2}}&0\\
\frac{1}{2}&0&\frac{1}{2}
\end{pmatrix}
\cdot
\begin{pmatrix}
-1&-\frac{1}{3}&-\frac{1}{3}&-1\\
0&0&\frac{1}{3a}&\frac{1}{2a}\\
1&1&1&1
\end{pmatrix}
\end{gather}

\section{Abbildungsunterschiede von Rotationen um ein Projektionszentrum und Rotation um einen beliebigen Drehpunkt von Punkten in der Ebene}

Bis jetzt wurden die Kameras jeweils nur um ihr Projektionszentrum gedreht, die Frage die jetzt noch im folgenden beantwortet werden soll ist, ob es ebenfalls möglich ist eine Homographiematrix für korrespondierende Punktepaare in einer Ebene aufzustellen, wenn eine Kamera um einen definierten Drehpunkt außerhalb der Kamera gedreht wurde. Um diesen Fall zu testen, muss erst einmal geklärt werden, inwieweit sich das Abbgebildete Bild in den Kameras ändert. Hierfür wurde eine kleine Simulation geschrieben, welche die Abbildung eines Objekts zeigt, wenn sich die Kamera um ihr Kamerazentrum dreht und im Vergleich hierzu wenn sie sich um einen definierten Drehpunkt dreht. In diesem Beispiel ist der rote Punkte in der Mitte des Quadrats der Drehpunkt, vergleiche Abbildung 3.3. Des Weiteren sollte geklärt werden, ob Punkte die bei der Drehung um das Projektionszentrum verdeckt bleiben auch bei einer Drehung um einen außerhalb der Kamera platzierten Drehpunkt ebenfalls verdeckt bleiben und ob eine gültige Homographiematrix errechnet werden kann. Abbildung 3.3 zeigt ein Quadrat mit vier Eckpunkten und seinem Mittelpunkt in rot.

\begin{minipage}{\linewidth}
	\centering
	\includegraphics[width=.8\linewidth]{images/Ausgangslage.png}
	\captionof{figure}{Objekt im Raum}
\end{minipage}\\


Für die Simulation der Drehung wurden zwei Schieberegler implementiert mit dem sich die Kamera einmal um ihre eigene y-Achse, in diesem Fall das Projektionszentrum und einmal um den Drehpunkt, welcher wie bereits gesagt der rote Mittelpunkt ist, drehen lässt.


\begin{minipage}{\linewidth}
	\centering
	\includegraphics[width=1.\linewidth]{images/DrehungPZ.png}
	\captionof{figure}{Drehung um das Projektionszentrum}
\end{minipage}\\ \\

Abblidung 3.4 zeigt jeweils die entstehende Abbildung wenn die Kamera um \ensuremath{20^\circ} beziehungsweise \ensuremath{-20^\circ} um das Projektionszentrum gedreht wurde.

\begin{minipage}{\linewidth}
	\centering
	\includegraphics[width=1.\linewidth]{images/DrehungDZ.png}
	\captionof{figure}{Drehung um einen Drehpunkt. In diesem Beispiel wurde der rote Punkt als Drehpunkt verwendet}
\end{minipage}\\ \\

Abbildung 3.5 zeigt die entstehenden Bilder, wenn die Kamera um \ensuremath{45^\circ} beziehungsweise \ensuremath{-45^\circ} um den Drehpunkt gedreht wurde. Wie sich zeigt ist hier ein weiterer grüner Punkt zu sehen welcher zu Testzwecken hinter dem roten Punkt platziert wurde sichtbar. Punkte die bei einer Drehung um das Projektionszentrum verdeckt bleiben, werden also bei einer Drehung um einen externen Drehpukt sichtbar. Graphisch kann man sich das folgendermaßen veranschaulichen.

\begin{minipage}{\linewidth}
	\centering
	\includegraphics[width=.8\linewidth]{images/GrafikDrehungProjektionszentrum.png}
	\captionof{figure}{Strahlengang durch das Projektionszentrum. Auf der Grafik ist erkennbar, dass der grüne Punkt auch nach der Drehung der Kamera vom roten Pukt verdeckt bleibt}
\end{minipage}\\ \\

\begin{minipage}{\linewidth}
	\centering
	\includegraphics[width=.8\linewidth]{images/GrafikDrehungUmDrehpunkt.png}
	\captionof{figure}{Strahlengang durch das Projektionszentrum. Auf der Grafik ist erkennbar, dass der grüne Punkt nach der Drehung der Kamera um einen Drehpunkt, welcher in diesem Fall der rote Punkt darstellt, sichtbar wird.}
\end{minipage}\\ \\

Die entstehenden Abbildungen sind also voneinander verschieden, jedoch liegen die fünf Punkte des Quadrates und dessen Mittelpunkt immer noch in einer Ebene. Die Behauptung die also nun noch zu beweisen gilt ist, dass sich für die jeweiligen Abbildungen beider Drehungen jeweils eine Homographiematrix $H$ aufstellen lässt, welche die Bilder der nicht-gedrehten und gedrehten Kameras ineinander überführen lässt. Die Homographiematrix soll eine Punkteüberführung des Ausgangsbildes, in Abblidung 3.6 und 3.7 also jeweils das linke Bild, in die gedrehten Bilder, jeweils die rechten Abbildungen, möglich machen. Das die Drehung um das Projektionszentrum eine gültige Homographiematrix hervorbringt wurde im vorherigen Unterkapitel (Link Kapitel 3.1) bereits gezeigt. Die Herangehensweise für das Beispiel um einen Drehpunkt ist größtenteils die selbe. Das einzige was sich unterscheidet ist die Transformation der Punkte in as Koordinatensystem von der gedrehten Kamera zwei. Diese besteht hier nicht nur aus einer beziehungsweise zwei hintereinander geschalteten Rotation, sondern zu aller erst muss die Kamera zwei zum gewünschten Drehpunkt verschoben, dann gedreht und zum Schluss wieder um den selben Translatationvektor zurück verschoben werden. Die Transformationsmatrix $M$ besteht dann aus drei Transformationsmatrizen \ensuremath{M = V_1 \cdot R \cdot V_2}. Matrix $M$ wird dann dazu verwendet, um die neue Position von Kamera zwei zu ermitteln. $O_c = [0 0 0 1]$ sind die Koordinaten von Kamera eins in Weltkoordinaten. In diesem Beispiel wird das Weltkoordinatensystem mit dem Kamerakoordinatensystem deckungsgleich definiert, so dass die Translationsmatrix $T_1$ von Kamera eins gleich der Identitätsmatrix ist. Mit $O_c2$ wird der Ursprung des Koordinatensystems von Kamera zwei bezeichnet. Es werden dies selben Punkt $A,B,C,D$ und $E$ wie im ersten Beispiel verwendet.\\

\begin{gather}
(A)_K=\begin{pmatrix}
0\\0\\2\\1
\end{pmatrix}, 
(B)_K=
\begin{pmatrix}
1\\0\\2\\1
\end{pmatrix},
(C)_K=
\begin{pmatrix}
1\\1\\2\\1
\end{pmatrix},
(D)_K=
\begin{pmatrix}
0\\1\\2\\1
\end{pmatrix},
(E)_K=
\begin{pmatrix}
\frac{1}{2}\\\frac{1}{2}\\2\\1
\end{pmatrix}
\end{gather}\\

Punkt $E$ bildet den Mittelpunkt des Quadrates und wird als $PivotPoint$ gewählt. Nun werden die drei Matrizen $M1, M2$ und $M3$ gebildet. $M1$ beinhaltet die Verschiebung von des Ursprungs von Kamera eins zum $PivotPoint$, $M2$ bildet die Rotationsmatrize, welche den Verschobenen Punkt um die gewünschten $45^\circ$ dreht. Die letzte Matrize $M3$ beinhaltet wieder eine Translation, welche den Punkt um den selben Wert wie die Translation zum $PivotPoint$ zurück verschiebt. 

\begin{gather}
	M1 = \begin{bmatrix}
	1&0&0&-\textit{PivotPoint}_x\\
	0&1&0&-\textit{PivotPoint}_y\\
	0&0&1&-\textit{PivotPoint}_z\\
	0&0&0&1
	\end{bmatrix} = 
	\begin{bmatrix}
	1&0&0&-\frac{1}{2}\\
	0&1&0&-\frac{1}{2}\\
	0&0&1&-2\\
	0&0&0&1
	\end{bmatrix}\\
	M2 = \begin{bmatrix}
	\cos(45^\circ)&0&\sin(45^\circ)&0\\
	0&1&0&0\\
	-\sin(45^\circ)&0&\cos(45^\circ)&0\\
	0&0&0&1
	\end{bmatrix}=
	\begin{bmatrix}
	\frac{1}{\sqrt{2}}&0&\frac{1}{\sqrt{2}}&0\\
	0&1&0&0\\
	-\frac{1}{\sqrt{2}}&0&\frac{1}{\sqrt{2}}&0\\
	0&0&0&1
	\end{bmatrix}\\
	M3 = 
	\begin{bmatrix}
	1&0&0&\textit{PivotPoint}_x\\
	0&1&0&\textit{PivotPoint}_y\\
	0&0&1&\textit{PivotPoint}_z\\
	0&0&0&1
	\end{bmatrix} = 
	\begin{bmatrix}
	1&0&0&\frac{1}{2}\\
	0&1&0&\frac{1}{2}\\
	0&0&1&2\\
	0&0&0&1
	\end{bmatrix}\\
	M= 
	M3 \cdot M2 \cdot M1
	= 
	\begin{bmatrix}
	\frac{1}{\sqrt{2}}&0&\frac{1}{\sqrt{2}}&-1.26777\\
	0&1&0&0\\
	-\frac{1}{\sqrt{2}}&0&\frac{1}{\sqrt{2}}&0.93934\\
	0&0&0&1
	\end{bmatrix}\\
	O_c2 = M \cdot O_c
	 = 	
	 \begin{bmatrix}
	\frac{1}{\sqrt{2}}&0&\frac{1}{\sqrt{2}}&-1.26777\\
	0&1&0&0\\
	-\frac{1}{\sqrt{2}}&0&\frac{1}{\sqrt{2}}&0.93934\\
	0&0&0&1
	\end{bmatrix} \cdot 
	\begin{bmatrix}
	0\\0\\0\\1
	\end{bmatrix}
	=
	\begin{bmatrix}
	-1.27\\0\\0.94\\1
	\end{bmatrix}	
\end{gather} 

Der Ursprung des Koordinatensystems von Kamera zwei befindet sich, angegeben in Weltkoordinaten, also bei $O_c2 =	\begin{bmatrix}-1.26777&0&0.93934&1\end{bmatrix}^T$. Mit dieser Information können die Punkte wieder in das Kamerakoordinatensystem von Kamera zwei transformiert werden. Es muss wieder eine Transformationsmatrit der Form 

\begin{gather}
	D = 	\begin{bmatrix}
	&&&\\
	&[C]^{-1}&& -[C]^{-1} \cdot V\\
	&&&\\
	0&0&0&1\\
	\end{bmatrix}
\end{gather}

aufgestellt werden. $V$ ist der Translationsvektor welcher den Wert von $O_c2$ annimmt. Da es sich wieder um kartesische Koordinatensysteme handelt gilt wieder $[C]^{-1}$ = $[C]^{T}$.

\begin{gather}
	 -[C]^{T}\cdot Oc_2 = 
	 \begin{pmatrix}
	 \frac{1}{\sqrt{2}}&0&-\frac{1}{\sqrt{2}}\\
	 0&1&0\\
	 \frac{1}{\sqrt{2}}&0&\frac{1}{\sqrt{2}}
	 \end{pmatrix}
	 \cdot
	 \begin{pmatrix}
	 -1.27\\0\\0.94
	 \end{pmatrix}
	 =
	  \begin{pmatrix}
	 1.56\\0\\0.23
	 \end{pmatrix}
\end{gather}

Für die Transformationsmatrizen $T_1$ und $T_2$ ergeben sich dann 

\begin{gather}
	T_1 = 
	\begin{bmatrix}
	1&0&0&0\\
	0&1&0&0\\
	0&0&1&0\\
	0&0&0&1
	\end{bmatrix}\\	
	T_2=
	 \begin{bmatrix}
	\frac{1}{\sqrt{2}}&0&-\frac{1}{\sqrt{2}}&1.56\\
	0&1&0&0\\
	\frac{1}{\sqrt{2}}&0&\frac{1}{\sqrt{2}}&0.23\\
	0&0&0&1
	\end{bmatrix}
\end{gather}\\

Nachdem $T_1$ und $T_2$ bestimmt sind, kann wie im vorherigen Beispiel weiter verfahren werden. Jetzt müssen die Punkte in Weltkoordinaten noch in die entsprechenden Kamerakoordinaten transformiert werden und danach noch mit den Projektionsmatrizen, welche für dieses Beispiel die selben bleiben, verrechnet werden.

 \begin{gather}
 \leftidx{_{K_{c1}}}{\begin{bmatrix}
 	\pi
 	\end{bmatrix}}{_{K_{c1}}}
 =		\leftidx{_{K_{c2}}}{\begin{bmatrix}
 	\pi
 	\end{bmatrix}}{_{K_{c2}}}
 =
 \begin{pmatrix}
 \zeta&0&0&0\\
 0&\zeta&0&0\\
 0&0&\zeta&0\\
 0&0&1&0
 \end{pmatrix}=
 \begin{pmatrix}
 -1&0&0&0\\
 0&-1&0&0\\
 0&0&-1&0\\
 0&0&1&0
 \end{pmatrix}
 \end{gather}
 
 Es entstehen die folgenden beiden Punktematrizen $pO_c$ für Kamera eins und $pO_c2$ für Kamera zwei. Die Koordinaten der jeweiligen Punkte $A,B,C,D$ und $E$ aus Sicht der beiden Kameras befinden sich der Reihe nach in den Spalten der Punktematrix. 
 
 \begin{gather}
 	pO_c = 
 	\begin{bmatrix}
 	0&\frac{1}{2}&\frac{1}{2}&0&\frac{1}{4}\\
 	0&0&\frac{1}{2}&\frac{1}{2}&\frac{1}{4}\\
 	1&1&1&1&1
 	\end{bmatrix}\\
 	pO_c2=
 		\begin{bmatrix}
 	0.09&0.36&0.36&0.09&\frac{1}{4}\\
 	0&0&0.42&0.61&\frac{1}{4}\\
 	1&1&1&1&1
 	\end{bmatrix}\\
 \end{gather}

Die Homographiematrix wird wie im Beispiel davor durch aufstellen der Koeffizientenmatrix. Da es sich auch hier nicht um einen überbestimmten Fall handelt, kann die homographiematrix entweder durch die Bestimmung des Kerns oder durch anwenden der Singulärwertszerlegung $SVD$, gewonnen werden. Die resultierende Homographiematrix sieht folgendermaßen aus.

\begin{gather}
	H = \begin{bmatrix}
	0.43&0&0.43\\
	0&0.6&0\\
	0.4&0&0.5
	\end{bmatrix}
\end{gather}

Wird $H$ auf die Punkte $pO_c$ angewandt, so liefert das Ergebnis die Punkte von $pO_c2$ und wird die Inverse $H^{-^1}$ auf die Punkte $pO_c2$ angewandt, so erhält man die Punkte von $pO_c$. Somit wurde bewiesen, dass Homographiematrizen immer zur Transformation von Punkte genutzt werden können, solange sich diese Punkte auf einer Ebene befinden. Die Definition der Homographie sagt aus, dass sowohl Tranlationen und Rotationen in der Homographiematrix vorkommen dürfen \cite{Roser} \cite{Peiffer}. Die Drehung um einen Drehpunkt ist nicht weiter als die Hintereinadnerschaltung verschiedener Tranformationsmarizen. 

(GRAPHIK IN GEOGEBRA ERSTELLEN) 


 Sollte wie im zweiten Beispiel der grüne Punkte mit durch eine Homographie übermittelt werden, so ist die entstehende Homographie ungültig. Kamera zwei ist in diesem Beispiel um ihr Projektionszentrum gedreht.\\

-> erklären warum homographien nur in der Ebene funktionieren
(GUTE ÜBERLEITUNG SCHAFFEN)


\section{Punkte in unterschiedlichen Ebenen}

Überleitung zur Epipolargeometry.\\
Warum kann hier keine Homographie benötigt werden\\
was muss hier genutzt werden?

\section{Epipolargeometrie als Grundlage der Stereokalibrierung und Szenenrekonstrunktion}


(MEHR LITERATUR FINDEN ZUM VERGLEICHEN)

\section{Geometrische Erläuterung der Fundamentalmatrix und der Essentiellen Matrix }







