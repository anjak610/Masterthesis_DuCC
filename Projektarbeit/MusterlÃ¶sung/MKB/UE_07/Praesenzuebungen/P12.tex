%MKB
%Mathematik
%Übungseinheit 7
%Hausübungen
%Aufgabe P12 

\setcounter{P-section}{12}
\renewcommand*\thesection{P\Nummerierung{\arabic{P-section}}}
\section{Berechnung einer Parallelprojektion- Schrägprojektion}

a) \\
\ensuremath{O=(0,0,0)}, \ensuremath{E_2=(0,1,0)}, \ensuremath{E_3 = (0,0,1)}

Berechnung der Richtungsvektoren \ensuremath{\overline{OE_2}} und \ensuremath{\overline{OE_3}}

\begin{gather}
	\overline{OE_2} = 
	\begin{pmatrix}
	0-0\\
	0-1\\
	0-0
	\end{pmatrix}
	= \begin{pmatrix}
	0\\-1\\0
	\end{pmatrix}\\
		\overline{OE_3} = 
	\begin{pmatrix}
	0-0\\
	0-0\\
	0-1
	\end{pmatrix}
	= \begin{pmatrix}
	0\\0\\-1
	\end{pmatrix}
	\end{gather}\\
	
Auufstellen der Hesse'schen Normalengleichung\\
	
	\begin{gather}
	\vec{n_0} \cdot [\vec{x}-\vec{p}]=0\\
	\vec{n_0}= \overline{OE_2} \times \overline{OE_3}=	
	\begin{pmatrix}
	0\\-1\\0
	\end{pmatrix}
	\times
	\begin{pmatrix}
	0\\0\\-1
	\end{pmatrix}
	=
	\begin{pmatrix}
	1\\0\\0
	\end{pmatrix}\\
		\vec{n_0} \cdot [\vec{x}-\vec{p}]=0\\ \leadsto 
		\begin{pmatrix}
		1\\0\\0
		\end{pmatrix} \cdot
		\begin{bmatrix}
		\begin{pmatrix}
		x_1\\x_2\\x_3
		\end{pmatrix}
		-
		\begin{pmatrix}
		0\\1\\0
		\end{pmatrix}
		\end{bmatrix}
		=0
\end{gather}\\

Aufpunkt \ensuremath{\vec{p}} kann hierbei sowohl  \ensuremath{E_2=(0,1,0)} als auch \ensuremath{E_3 = (0,0,1)} sein.\\

b) Richtungsvektor der Projektionsgeraden \ensuremath{\vec{r}=\begin{pmatrix}
		1\\1\\1
\end{pmatrix}}, \ensuremath{E_1 =(1,0,0)}\\

Für jeden Punkt \ensuremath{O,E_1,E_2,E_3} muss eine Parametergleichung der Form \ensuremath{\vec{x} = \vec{a}+t\cdot \vec{r}} aufgestellt werden.

\begin{gather}
O\coloneqq \vec{x}=\begin{pmatrix}
0\\0\\0
\end{pmatrix} +t\cdot\begin{pmatrix}
1\\1\\1
\end{pmatrix}\\
E_1\coloneqq \vec{x}=\begin{pmatrix}
1\\0\\0
\end{pmatrix} +t\cdot\begin{pmatrix}
1\\1\\1
\end{pmatrix}\\
E_2\coloneqq \vec{x}=\begin{pmatrix}
0\\1\\0
\end{pmatrix} +t\cdot\begin{pmatrix}
1\\1\\1
\end{pmatrix}\\
E_3\coloneqq \vec{x}=\begin{pmatrix}
0\\0\\1
\end{pmatrix} +t\cdot\begin{pmatrix}
1\\1\\1
\end{pmatrix}
\end{gather}\\

Diese Parametergleichungen werden nun in der Hesse'schen Normalform der Ebene anstelle von \ensuremath{\vec{x}} eingesetzt.

\begin{gather}
	\vec{n_0} \cdot (\vec{a}+t\cdot \vec{r}-\vec{p})=0\\
	\leadsto 
	\begin{pmatrix}
	1\\0\\0
	\end{pmatrix} \cdot 
	\begin{bmatrix}
	\begin{pmatrix}
	0\\0\\0
	\end{pmatrix}
	+ t
	\begin{pmatrix}
	1\\1\\1
	\end{pmatrix}
	-
	\begin{pmatrix}
	0\\1\\0
	\end{pmatrix}
	\end{bmatrix} = 0\\
	\leadsto
	\begin{pmatrix}
	1\\0\\0
	\end{pmatrix}\cdot
	\begin{pmatrix}
	0\\0\\0
	\end{pmatrix}
	+
	\begin{pmatrix}
	1\\0\\0
	\end{pmatrix}
	\cdot
	\begin{pmatrix}
	t\\t\\t
	\end{pmatrix}
	-
	\begin{pmatrix}
	1\\0\\0
	\end{pmatrix}
	\cdot
	\begin{pmatrix}
	0\\1\\0
	\end{pmatrix} =0\\
	0+0+0+t+0+0-0+0+0=0\\
	\leadsto t=0
\end{gather}

\ensuremath{t} einsetzen in \ensuremath{O:= \vec{x}=\begin{pmatrix}
		0\\0\\0
	\end{pmatrix} +t\cdot\begin{pmatrix}
		1\\1\\1
\end{pmatrix}}.\\

\begin{gather}
	O'\coloneqq \vec{x} = \begin{pmatrix}
	0\\0\\0
	\end{pmatrix}+0 \begin{pmatrix}
		1\\1\\1
	\end{pmatrix} 
\leadsto \vec{x} =
 \begin{pmatrix}
	0\\0\\0
\end{pmatrix}
\end{gather}\\

Für \ensuremath{E_1}

\begin{gather}
\vec{n_0} \cdot (\vec{a}+t\cdot \vec{r}-\vec{p})=0\\
\leadsto 
\begin{pmatrix}
1\\0\\0
\end{pmatrix} \cdot 
\begin{bmatrix}
\begin{pmatrix}
1\\0\\0
\end{pmatrix}
+ t
\begin{pmatrix}
1\\1\\1
\end{pmatrix}
-
\begin{pmatrix}
0\\1\\0
\end{pmatrix}
\end{bmatrix} = 0\\
\leadsto
\begin{pmatrix}
1\\0\\0
\end{pmatrix}\cdot
\begin{pmatrix}
1\\0\\0
\end{pmatrix}
+
\begin{pmatrix}
1\\0\\0
\end{pmatrix}
\cdot
\begin{pmatrix}
t\\t\\t
\end{pmatrix}
-
\begin{pmatrix}
1\\0\\0
\end{pmatrix}
\cdot
\begin{pmatrix}
0\\1\\0
\end{pmatrix} =0\\
1+0+0+t+0+0-0+0+0=0\\
\leadsto t=-1
\end{gather}

\ensuremath{t} einsetzen in \ensuremath{E_1:= \vec{x}=\begin{pmatrix}
		1\\0\\0
	\end{pmatrix} +t\cdot\begin{pmatrix}
		1\\1\\1
\end{pmatrix}}.\\

\begin{gather}
E_1'\coloneqq \vec{x} = \begin{pmatrix}
1\\0\\0
\end{pmatrix}+(-1) \begin{pmatrix}
1\\1\\1
\end{pmatrix} 
\leadsto \vec{x} =
\begin{pmatrix}
0\\-1\\-1
\end{pmatrix}
\end{gather}\\

Für \ensuremath{E_2}

\begin{gather}
\vec{n_0} \cdot (\vec{a}+t\cdot \vec{r}-\vec{p})=0\\
\leadsto 
\begin{pmatrix}
1\\0\\0
\end{pmatrix} \cdot 
\begin{bmatrix}
\begin{pmatrix}
0\\1\\0
\end{pmatrix}
+ t
\begin{pmatrix}
1\\1\\1
\end{pmatrix}
-
\begin{pmatrix}
0\\1\\0
\end{pmatrix}
\end{bmatrix} = 0\\
\leadsto
\begin{pmatrix}
1\\0\\0
\end{pmatrix}\cdot
\begin{pmatrix}
0\\1\\0
\end{pmatrix}
+
\begin{pmatrix}
1\\0\\0
\end{pmatrix}
\cdot
\begin{pmatrix}
t\\t\\t
\end{pmatrix}
-
\begin{pmatrix}
1\\0\\0
\end{pmatrix}
\cdot
\begin{pmatrix}
0\\1\\0
\end{pmatrix} =0\\
0+0+0+t+0+0-0+0+0=0\\
\leadsto t=0
\end{gather}

\ensuremath{t} einsetzen in \ensuremath{E_2:= \vec{x}=\begin{pmatrix}
		0\\1\\0
	\end{pmatrix} +t\cdot\begin{pmatrix}
		1\\1\\1
\end{pmatrix}}.\\

\begin{gather}
E_2'\coloneqq \vec{x} = \begin{pmatrix}
0\\1\\0
\end{pmatrix}+(0) \begin{pmatrix}
1\\1\\1
\end{pmatrix} 
\leadsto \vec{x} =
\begin{pmatrix}
0\\1\\0
\end{pmatrix}
\end{gather}\\

Für \ensuremath{E_3}

\begin{gather}
\vec{n_0} \cdot (\vec{a}+t\cdot \vec{r}-\vec{p})=0\\
\leadsto 
\begin{pmatrix}
1\\0\\0
\end{pmatrix} \cdot 
\begin{bmatrix}
\begin{pmatrix}
0\\0\\1
\end{pmatrix}
+ t
\begin{pmatrix}
1\\1\\1
\end{pmatrix}
-
\begin{pmatrix}
0\\1\\0
\end{pmatrix}
\end{bmatrix} = 0\\
\leadsto
\begin{pmatrix}
1\\0\\0
\end{pmatrix}\cdot
\begin{pmatrix}
0\\0\\1
\end{pmatrix}
+
\begin{pmatrix}
1\\0\\0
\end{pmatrix}
\cdot
\begin{pmatrix}
t\\t\\t
\end{pmatrix}
-
\begin{pmatrix}
1\\0\\0
\end{pmatrix}
\cdot
\begin{pmatrix}
0\\1\\0
\end{pmatrix} =0\\
0+0+0+t+0+0-0+0+0=0\\
\leadsto t=0
\end{gather}

\ensuremath{t} einsetzen in \ensuremath{E_3:= \vec{x}=\begin{pmatrix}
		0\\0\\1
	\end{pmatrix} +t\cdot\begin{pmatrix}
		1\\1\\1
\end{pmatrix}}.\\

\begin{gather}
E_3'\coloneqq \vec{x} = \begin{pmatrix}
0\\0\\1
\end{pmatrix}+(0) \begin{pmatrix}
1\\1\\1
\end{pmatrix} 
\leadsto \vec{x} =
\begin{pmatrix}
0\\0\\1
\end{pmatrix}
\end{gather}\\

c)\\


\begin{tikzpicture}[x=4cm,y=4cm,z=4cm,>=stealth]

\draw[step=4cm,lightgray,very thin,dashed] (-1.5,1.5) grid (1.5,-1.5);

% The axes
\draw[->] (xyz cs:x=-1.5) -- (xyz cs:x=1.5) node[above] {$x2$};
\draw[->] (xyz cs:y=-1.5) -- (xyz cs:y=1.5) node[right] {$x3$};
\draw[->] (xyz cs:z=-1.5) -- (xyz cs:z=1.5) node[above] {$x1$};

\foreach \coo in {1}
{
	\draw[thick] (\coo,-1pt) -- (\coo,1pt) node[below=1pt] {\coo};
	\draw[thick] (-1pt,\coo) -- (1pt,\coo) node[left=1pt] {\coo};
	\draw[thick] (xyz cs:y=-0.05pt,z=-\coo) -- (xyz cs:y=0.05pt,z=-\coo) node[above] {\coo};
}

		
\fill[red](xyz cs: x=0,y=0,z=0) circle(3pt)node[above]{$O$};
\fill[red](xyz cs: x=-1,y=-1,z=0) circle(3pt)node[above]{$E1$};
\fill[red](xyz cs: x=1,y=0,z=0) circle(3pt)node[above]{$E2$};
\fill[red](xyz cs: x=0,y=1,z=0) circle(3pt)node[above]{$E3$};

\end{tikzpicture}\\


d) \ensuremath{s_1 = ||\overrightarrow{OE_1'}||}, \ensuremath{s_2 = ||\overrightarrow{OE_2'}||}, \ensuremath{s_3 = ||\overrightarrow{OE_3'}||}, \ensuremath{\alpha  \angle (\overrightarrow{OE_1'},\overrightarrow{OE_3'})}, \ensuremath{\beta  \angle (\overrightarrow{OE_2'},\overrightarrow{OE_3'})}  

\begin{gather}
	s_1 = \sqrt{1^2+0^2+0^2} \leadsto s_1 = 1\\
	s_2 = \sqrt{0^2+1^2+0^2} \leadsto s_1 = 1\\
	s_1 = \sqrt{0^2+(-1)^2+(-1)^2} \leadsto s_1 = \sqrt{2}\\
	\cos(\alpha)= \frac{\overrightarrow{OE_1'} \cdot \overrightarrow{OE_3'}}{|\overrightarrow{OE_1'}|\cdot |\overrightarrow{OE_3'}|} 
	\leadsto
		\cos(\alpha)= \frac{\begin{pmatrix}
		0\\-1\\-1
		\end{pmatrix}\cdot \begin{pmatrix}
		0\\0\\1
		\end{pmatrix}}{\sqrt{2} \cdot 1}
	\leadsto
		\cos(\alpha)= \frac{0+0-1}{\sqrt{2}}\\ \leadsto 	\cos(\alpha)= -\frac{1}{\sqrt{2}} \leadsto \cos^{-1}(-\frac{1}{\sqrt{2}}) = 135^{\circ}\\		
	\cos(\beta)= \frac{\overrightarrow{OE_2'} \cdot \overrightarrow{OE_3'}}{|\overrightarrow{OE_2'}|\cdot |\overrightarrow{OE_3'}|} 
		\leadsto
		\cos(\beta)= \frac{\begin{pmatrix}
			0\\1\\0
			\end{pmatrix}\cdot \begin{pmatrix}
			0\\0\\1
			\end{pmatrix}}{1 \cdot 1}
	\leadsto
		\cos(\beta)= \frac{0+0+0}{1}\\ \leadsto  \cos^{-1}(0) = 90^{\circ}	
\end{gather}

e) \ensuremath{A = (1,1,0)}, \ensuremath{B=(1,0,1)}, \ensuremath{C=(0,1,1)}, \ensuremath{ D=(1,1,1)}, Bildebene: \ensuremath{\vec{n_0} \cdot (\vec{x}-\vec{p})=0}, 
\ensuremath{\vec{n_0} = \begin{pmatrix}
		1\\0\\0
\end{pmatrix}}, Projektionsrichtung: \ensuremath{r = \begin{pmatrix}
1\\1\\1
\end{pmatrix}}

Für jeden Punkt \ensuremath{A,B,C,D} muss eine Parametergleichung der Form \ensuremath{\vec{x} = \vec{a}+t\cdot \vec{r}} aufgestellt werden.

\begin{gather}
A\coloneqq \vec{x}=\begin{pmatrix}
1\\1\\0
\end{pmatrix} +t\cdot\begin{pmatrix}
1\\1\\1
\end{pmatrix}\\
B\coloneqq \vec{x}=\begin{pmatrix}
1\\0\\1
\end{pmatrix} +t\cdot\begin{pmatrix}
1\\1\\1
\end{pmatrix}\\
C\coloneqq \vec{x}=\begin{pmatrix}
0\\1\\1
\end{pmatrix} +t\cdot\begin{pmatrix}
1\\1\\1
\end{pmatrix}\\
D\coloneqq \vec{x}=\begin{pmatrix}
1\\1\\1
\end{pmatrix} +t\cdot\begin{pmatrix}
1\\1\\1
\end{pmatrix}
\end{gather}\\

 
Für \ensuremath{A}  %Ab hier weiter schreiben
 
\begin{gather}
\vec{n_0} \cdot (\vec{a}+t\cdot \vec{r}-\vec{p})=0\\
\leadsto 
\begin{pmatrix}
1\\0\\0
\end{pmatrix} \cdot 
\begin{bmatrix}
\begin{pmatrix}
1\\1\\0
\end{pmatrix}
+ t
\begin{pmatrix}
1\\1\\1
\end{pmatrix}
-
\begin{pmatrix}
0\\1\\0
\end{pmatrix}
\end{bmatrix} = 0\\
\leadsto
\begin{pmatrix}
1\\0\\0
\end{pmatrix}\cdot
\begin{pmatrix}
1\\1\\0
\end{pmatrix}
+
\begin{pmatrix}
1\\0\\0
\end{pmatrix}
\cdot
\begin{pmatrix}
t\\t\\t
\end{pmatrix}
-
\begin{pmatrix}
1\\0\\0
\end{pmatrix}
\cdot
\begin{pmatrix}
0\\1\\0
\end{pmatrix} =0\\
1+0+0+t+0+0-0+0+0=0\\
\leadsto t=-1
\end{gather}

\ensuremath{t} einsetzen in \ensuremath{A':= \vec{x}=\begin{pmatrix}
		1\\1\\0
	\end{pmatrix} +t\cdot\begin{pmatrix}
		1\\1\\1
\end{pmatrix}}.\\

\begin{gather}
A'\coloneqq \vec{x} = \begin{pmatrix}
1\\1\\0
\end{pmatrix}+(-1) \begin{pmatrix}
1\\1\\1
\end{pmatrix} 
\leadsto \vec{x} =
\begin{pmatrix}
0\\0\\-1
\end{pmatrix}
\end{gather}\\


Für \ensuremath{B}  %Ab hier weiter schreiben

\begin{gather}
\vec{n_0} \cdot (\vec{a}+t\cdot \vec{r}-\vec{p})=0\\
\leadsto 
\begin{pmatrix}
1\\0\\0
\end{pmatrix} \cdot 
\begin{bmatrix}
\begin{pmatrix}
1\\0\\1
\end{pmatrix}
+ t
\begin{pmatrix}
1\\1\\1
\end{pmatrix}
-
\begin{pmatrix}
0\\1\\0
\end{pmatrix}
\end{bmatrix} = 0\\
\leadsto
\begin{pmatrix}
1\\0\\0
\end{pmatrix}\cdot
\begin{pmatrix}
1\\0\\1
\end{pmatrix}
+
\begin{pmatrix}
1\\0\\0
\end{pmatrix}
\cdot
\begin{pmatrix}
t\\t\\t
\end{pmatrix}
-
\begin{pmatrix}
1\\0\\0
\end{pmatrix}
\cdot
\begin{pmatrix}
0\\1\\0
\end{pmatrix} =0\\
1+0+0+t+0+0-0+0+0=0\\
\leadsto t=-1
\end{gather}

\ensuremath{t} einsetzen in \ensuremath{B:= \vec{x}=\begin{pmatrix}
		1\\0\\1
	\end{pmatrix} +t\cdot\begin{pmatrix}
		1\\1\\1
\end{pmatrix}}.\\

\begin{gather}
B'\coloneqq \vec{x} = \begin{pmatrix}
1\\0\\1
\end{pmatrix}+(-1) \begin{pmatrix}
1\\1\\1
\end{pmatrix} 
\leadsto \vec{x} =
\begin{pmatrix}
0\\-1\\0
\end{pmatrix}
\end{gather}\\

Für \ensuremath{C}  %Ab hier weiter schreiben

\begin{gather}
\vec{n_0} \cdot (\vec{a}+t\cdot \vec{r}-\vec{p})=0\\
\leadsto 
\begin{pmatrix}
1\\0\\0
\end{pmatrix} \cdot 
\begin{bmatrix}
\begin{pmatrix}
0\\1\\1
\end{pmatrix}
+ t
\begin{pmatrix}
1\\1\\1
\end{pmatrix}
-
\begin{pmatrix}
0\\1\\0
\end{pmatrix}
\end{bmatrix} = 0\\
\leadsto
\begin{pmatrix}
1\\0\\0
\end{pmatrix}\cdot
\begin{pmatrix}
0\\1\\1
\end{pmatrix}
+
\begin{pmatrix}
1\\0\\0
\end{pmatrix}
\cdot
\begin{pmatrix}
t\\t\\t
\end{pmatrix}
-
\begin{pmatrix}
1\\0\\0
\end{pmatrix}
\cdot
\begin{pmatrix}
0\\1\\0
\end{pmatrix} =0\\
0+0+0+t+0+0-0+0+0=0\\
\leadsto t=0
\end{gather}

\ensuremath{t} einsetzen in \ensuremath{C:= \vec{x}=\begin{pmatrix}
		0\\1\\1
	\end{pmatrix} +t\cdot\begin{pmatrix}
		1\\1\\1
\end{pmatrix}}\\

\begin{gather}
C'\coloneqq \vec{x} = \begin{pmatrix}
0\\1\\1
\end{pmatrix}+(0) \begin{pmatrix}
1\\1\\1
\end{pmatrix} 
\leadsto \vec{x} =
\begin{pmatrix}
0\\1\\1
\end{pmatrix}
\end{gather}\\

Für \ensuremath{D}  %Ab hier weiter schreiben\\
\begin{gather}
\vec{n_0} \cdot (\vec{a}+t\cdot \vec{r}-\vec{p})=0\\
\leadsto 
\begin{pmatrix}
1\\0\\0
\end{pmatrix} \cdot 
\begin{bmatrix}
\begin{pmatrix}
1\\1\\1
\end{pmatrix}
+ t
\begin{pmatrix}
1\\1\\1
\end{pmatrix}
-
\begin{pmatrix}
0\\1\\0
\end{pmatrix}
\end{bmatrix} = 0\\
\leadsto
\begin{pmatrix}
1\\0\\0
\end{pmatrix}\cdot
\begin{pmatrix}
1\\1\\1
\end{pmatrix}
+
\begin{pmatrix}
1\\0\\0
\end{pmatrix}
\cdot
\begin{pmatrix}
t\\t\\t
\end{pmatrix}
-
\begin{pmatrix}
1\\0\\0
\end{pmatrix}
\cdot
\begin{pmatrix}
0\\1\\0
\end{pmatrix} =0\\
1+0+0+t+0+0-0+0+0=0\\
\leadsto t=-1
\end{gather} 
\ensuremath{t} einsetzen in \ensuremath{D:= \vec{x}=\begin{pmatrix}
		1\\1\\1
	\end{pmatrix} +t\cdot\begin{pmatrix}
		1\\1\\1
\end{pmatrix}}
\begin{gather}
D'\coloneqq \vec{x} = \begin{pmatrix}
1\\1\\1
\end{pmatrix}+(-1) \begin{pmatrix}
1\\1\\1
\end{pmatrix} 
\leadsto \vec{x} =
\begin{pmatrix}
0\\0\\0
\end{pmatrix}
\end{gather}

\pagebreak

f)

\begin{center}
\begin{tikzpicture}[x=4cm,y=4cm,z=4cm,>=stealth]


\draw[step=4cm,lightgray,very thin,dashed] (-1.5,1.5) grid (1.5,-1.5);

% The axes
\draw[->] (xyz cs:x=-1.5) -- (xyz cs:x=1.5) node[above] {$x3$};
\draw[->] (xyz cs:y=-1.5) -- (xyz cs:y=1.5) node[right] {$x2$};
\draw[->] (xyz cs:z=-1.5) -- (xyz cs:z=1.5) node[above] {$x1$};

\foreach \coo in {1}
{
	\draw[thick] (\coo,-1pt) -- (\coo,1pt) node[below=1pt] {\coo};
	\draw[thick] (-1pt,\coo) -- (1pt,\coo) node[left=1pt] {\coo};
	\draw[thick] (xyz cs:y=-0.05pt,z=-\coo) -- (xyz cs:y=0.05pt,z=-\coo) node[above] {\coo};
}


%\fill[red](xyz cs: x=0,y=0,z=0) circle(3pt)node[above]{$O$};
%\fill[red](xyz cs: x=-1,y=-1,z=0) circle(3pt)node[above]{$E1$};
%\fill[red](xyz cs: x=1,y=0,z=0) circle(3pt)node[above]{$E2$};
%\fill[red](xyz cs: x=0,y=1,z=0) circle(3pt)node[above]{$E3$};

\fill[blue](xyz cs: x=-1,y=0,z=0) circle(3pt)node[below]{$A'$};
\fill[blue](xyz cs: x=0,y=-1,z=0) circle(3pt)node[below]{$B'$};
\fill[blue](xyz cs: x=1,y=1,z=0) circle(3pt)node[below]{$C'$};
\fill[blue](xyz cs: x=0,y=0,z=0) circle(3pt)node[below]{$D'$};

\fill[green](xyz cs: x=-1,y=-1,z=0) circle(3pt)node[below]{$C''$};
\fill[green](xyz cs: x=0,y=0,z=0) circle(3pt)node[above]{$A''$};
\fill[green](xyz cs: x=1,y=0,z=0) circle(3pt)node[above]{$B''$};
\fill[green](xyz cs: x=-1,y=0,z=1) circle(3pt)node[below]{$D''$};
\end{tikzpicture}
\end{center}