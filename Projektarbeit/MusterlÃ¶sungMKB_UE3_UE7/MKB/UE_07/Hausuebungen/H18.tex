%MKB
%Mathematik 2
%Übungseinheit 7
%Hausübungen
%Aufgabe H18

\setcounter{H-section}{18}
\renewcommand*\thesection{H\Nummerierung{\arabic{H-section}}}
\section{Orthogonale isometrische Parallelprojektion}


a)\\

Der Normalenvektor ist \ensuremath{\vec{n_0} = (\begin{pmatrix}
		1\\1\\1
	\end{pmatrix})}. Um die Nachzurpüfen können wir das Kreizprodukt aus \ensuremath{\vec{u} := \overline{E_2E_3} = \begin{pmatrix}
	0\\-1\\1
\end{pmatrix} } und \ensuremath{\vec{v} := \overline{E_2E_1} = \begin{pmatrix}
1\\-1\\0
\end{pmatrix} }

\begin{gather}
	\vec{n_0} = \vec{u} \times \vec{v}\\
	\vec{n_0} = \begin{pmatrix}
	0\\-1\\1
	\end{pmatrix} \times 
	\begin{pmatrix}
	1\\-1\\0
	\end{pmatrix}\\
	\vec{n_0} = \begin{pmatrix}
	1\\1\\1
	\end{pmatrix}
\end{gather} \\

b) Projektionbilder von \ensuremath{E_1  = \begin{pmatrix}
		1\\0\\0
\end{pmatrix}, E_2 = \begin{pmatrix}
0\\1\\0
\end{pmatrix}, E_3 = \begin{pmatrix}
0\\0\\1
\end{pmatrix}}\\


\begin{gather}
O\coloneqq \vec{x}=\begin{pmatrix}
0\\0\\0
\end{pmatrix} +t\cdot\begin{pmatrix}
1\\1\\1
\end{pmatrix}\\
E_1\coloneqq \vec{x}=\begin{pmatrix}
1\\0\\0
\end{pmatrix} +t\cdot\begin{pmatrix}
1\\1\\1
\end{pmatrix}\\
E_2\coloneqq \vec{x}=\begin{pmatrix}
0\\1\\0
\end{pmatrix} +t\cdot\begin{pmatrix}
1\\1\\1
\end{pmatrix}\\
E_3\coloneqq \vec{x}=\begin{pmatrix}
0\\0\\1
\end{pmatrix} +t\cdot\begin{pmatrix}
1\\1\\1
\end{pmatrix}
\end{gather}\\

Für \ensuremath{E_1}  %Ab hier weiter schreiben

\begin{gather}
\vec{n_0} \cdot (\vec{a}+t\cdot \vec{r}-\vec{p})=0\\
\leadsto 
\begin{pmatrix}
1\\1\\1
\end{pmatrix} \cdot 
\begin{bmatrix}
\begin{pmatrix}
1\\0\\0
\end{pmatrix}
+ t
\begin{pmatrix}
1\\1\\1
\end{pmatrix}
-
\begin{pmatrix}
1\\0\\0
\end{pmatrix}
\end{bmatrix} = 0\\
\leadsto
\begin{pmatrix}
1\\1\\1
\end{pmatrix}\cdot
\begin{pmatrix}
1\\0\\0
\end{pmatrix}
+
\begin{pmatrix}
1\\1\\1
\end{pmatrix}
\cdot
\begin{pmatrix}
t\\t\\t
\end{pmatrix}
-
\begin{pmatrix}
1\\1\\1
\end{pmatrix}
\cdot
\begin{pmatrix}
1\\0\\0
\end{pmatrix} =0\\
1+0+0+t+t+t-1-0-0=0\\
\leadsto t=0
\end{gather}

\ensuremath{t} einsetzen in \ensuremath{E_1:= \vec{x}=\begin{pmatrix}
		0\\1\\1
	\end{pmatrix} +t\cdot\begin{pmatrix}
		1\\1\\1
\end{pmatrix}}\\

\begin{gather}
E_1'\coloneqq \vec{x} = \begin{pmatrix}
1\\0\\0
\end{pmatrix}+(0) \begin{pmatrix}
1\\1\\1
\end{pmatrix} 
\leadsto \vec{x} =
\begin{pmatrix}
1\\0\\0
\end{pmatrix}
\end{gather}\\

Für \ensuremath{E_2}  %Ab hier weiter schreiben

\begin{gather}
\vec{n_0} \cdot (\vec{a}+t\cdot \vec{r}-\vec{p})=0\\
\leadsto 
\begin{pmatrix}
1\\1\\1
\end{pmatrix} \cdot 
\begin{bmatrix}
\begin{pmatrix}
0\\1\\0
\end{pmatrix}
+ t
\begin{pmatrix}
1\\1\\1
\end{pmatrix}
-
\begin{pmatrix}
1\\0\\0
\end{pmatrix}
\end{bmatrix} = 0\\
\leadsto
\begin{pmatrix}
1\\1\\1
\end{pmatrix}\cdot
\begin{pmatrix}
0\\1\\0
\end{pmatrix}
+
\begin{pmatrix}
1\\1\\1
\end{pmatrix}
\cdot
\begin{pmatrix}
t\\t\\t
\end{pmatrix}
-
\begin{pmatrix}
1\\1\\1
\end{pmatrix}
\cdot
\begin{pmatrix}
1\\0\\0
\end{pmatrix} =0\\
1+0+0+t+t+t-1-0-0=0\\
\leadsto t=0
\end{gather}

\ensuremath{t} einsetzen in \ensuremath{E_2:= \vec{x}=\begin{pmatrix}
		0\\1\\0
	\end{pmatrix} +t\cdot\begin{pmatrix}
		1\\1\\1
\end{pmatrix}}\\

\begin{gather}
E_2'\coloneqq \vec{x} = \begin{pmatrix}
0\\1\\0
\end{pmatrix}+(0) \begin{pmatrix}
1\\1\\1
\end{pmatrix} 
\leadsto \vec{x} =
\begin{pmatrix}
0\\1\\0
\end{pmatrix}
\end{gather}\\

Für \ensuremath{E_3}  %Ab hier weiter schreiben

\begin{gather}
\vec{n_0} \cdot (\vec{a}+t\cdot \vec{r}-\vec{p})=0\\
\leadsto 
\begin{pmatrix}
1\\1\\1
\end{pmatrix} \cdot 
\begin{bmatrix}
\begin{pmatrix}
0\\0\\1
\end{pmatrix}
+ t
\begin{pmatrix}
1\\1\\1
\end{pmatrix}
-
\begin{pmatrix}
1\\0\\0
\end{pmatrix}
\end{bmatrix} = 0\\
\leadsto
\begin{pmatrix}
1\\1\\1
\end{pmatrix}\cdot
\begin{pmatrix}
0\\0\\1
\end{pmatrix}
+
\begin{pmatrix}
1\\1\\1
\end{pmatrix}
\cdot
\begin{pmatrix}
t\\t\\t
\end{pmatrix}
-
\begin{pmatrix}
1\\1\\1
\end{pmatrix}
\cdot
\begin{pmatrix}
1\\0\\0
\end{pmatrix} =0\\
0+0+1+t+t+t-1-0-0=0\\
\leadsto t=0
\end{gather}

\ensuremath{t} einsetzen in \ensuremath{E_3:= \vec{x}=\begin{pmatrix}
		0\\0\\1
	\end{pmatrix} +t\cdot\begin{pmatrix}
		1\\1\\1
\end{pmatrix}}\\

\begin{gather}
E_3'\coloneqq \vec{x} = \begin{pmatrix}
0\\0\\1
\end{pmatrix}+(0) \begin{pmatrix}
1\\1\\1
\end{pmatrix} 
\leadsto \vec{x} =
\begin{pmatrix}
0\\0\\1
\end{pmatrix}
\end{gather}\\


Für \ensuremath{O}  %Ab hier weiter schreiben

\begin{gather}
\vec{n_0} \cdot (\vec{a}+t\cdot \vec{r}-\vec{p})=0\\
\leadsto 
\begin{pmatrix}
1\\1\\1
\end{pmatrix} \cdot 
\begin{bmatrix}
\begin{pmatrix}
0\\0\\0
\end{pmatrix}
+ t
\begin{pmatrix}
1\\1\\1
\end{pmatrix}
-
\begin{pmatrix}
1\\0\\0
\end{pmatrix}
\end{bmatrix} = 0\\
\leadsto
\begin{pmatrix}
1\\1\\1
\end{pmatrix}\cdot
\begin{pmatrix}
0\\0\\0
\end{pmatrix}
+
\begin{pmatrix}
1\\1\\1
\end{pmatrix}
\cdot
\begin{pmatrix}
t\\t\\t
\end{pmatrix}
-
\begin{pmatrix}
1\\1\\1
\end{pmatrix}
\cdot
\begin{pmatrix}
1\\0\\0
\end{pmatrix} =0\\
0+0+0+t+t+t-1-0-0=0\\
\leadsto t= \frac{1}{3}
\end{gather}

\ensuremath{t} einsetzen in \ensuremath{O:= \vec{x}=\begin{pmatrix}
		0\\0\\0
	\end{pmatrix} +t\cdot\begin{pmatrix}
		1\\1\\1
\end{pmatrix}}\\

\begin{gather}
O'\coloneqq \vec{x} = \begin{pmatrix}
0\\0\\0
\end{pmatrix}+\ensuremath{\frac{1}{3}} \begin{pmatrix}
1\\1\\1
\end{pmatrix} 
\leadsto \vec{x} =
\begin{pmatrix}
\ensuremath{\frac{1}{3}}\\\ensuremath{\frac{1}{3}}\\\ensuremath{\frac{1}{3}}
\end{pmatrix}
\end{gather}\\

c) Bestimmen von \ensuremath{s_1 = \overline{O'E_1'},\, s_2 = \overline{O'E_2'},\, s_3 = \overline{O'E_3'}}.\\

\begin{gather}
	\overline{O'E_1'} = \begin{pmatrix}
	1-\frac{1}{3}\\0-\frac{1}{3}\\0-\frac{1}{3}
	\end{pmatrix} 
	=
	\begin{pmatrix}
	\frac{2}{3}\\-\frac{1}{3}\\-\frac{1}{3}
	\end{pmatrix}\\
	\leadsto
	|\overline{O'E_1'}| = \sqrt{{\frac{2}{3}}^2 -{\frac{1}{3}}^2 -{\frac{1}{3}}^2}\\
		\leadsto
	|\overline{O'E_1'}| = \frac{\sqrt{2}}{3}	
\end{gather}

\begin{gather}
\overline{O'E_2'} = \begin{pmatrix}
0-\frac{1}{3}\\1-\frac{1}{3}\\0-\frac{1}{3}
\end{pmatrix} 
=
\begin{pmatrix}
-\frac{1}{3}\\\frac{2}{3}\\-\frac{1}{3}
\end{pmatrix}\\
\leadsto
|\overline{O'E_2'}| = \sqrt{{-\frac{2}{3}}^2 +{\frac{2}{3}}^2 -{\frac{1}{3}}^2}\\
\leadsto
|\overline{O'E_2'}| = \frac{\sqrt{2}}{3}
\end{gather}

\begin{gather}
\overline{O'E3'} = \begin{pmatrix}
0-\frac{1}{3}\\0-\frac{1}{3}\\1-\frac{1}{3}
\end{pmatrix} 
=
\begin{pmatrix}
-\frac{1}{3}\\-\frac{1}{3}\\\frac{2}{3}
\end{pmatrix}\\
\leadsto
|\overline{O'E_3'}| = \sqrt{{-\frac{2}{3}}^2 -{\frac{1}{3}}^2 +{\frac{2}{3}}^2}\\
\leadsto
|\overline{O'E_3'}| = \frac{\sqrt{2}}{3}	
\end{gather}







